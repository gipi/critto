% TODO: fare si che ogni principale porzione del testo (sezioni, capitoli, appendici)
% abbiano il proprio namespace così da potere differenziare definizioni, teoremi, etc...
% in base al loro contesto
\input eplain
\input tikz
\beginpackages
  \usepackage{url}
\endpackages
%%%%%%%% SALVIAMO QUALCHE PARAMETRO COME È DI DEFAULT %%%%%%%%%%
\newdimen\defaultparskip\defaultparskip=\parskip
\newdimen\defaultparindent\defaultparindent=\parindent
%%%%%%%%%%%%%%%%%%%%%%%%%%%%%%%%%%%%%%%%%%%%%%%%%%%%%%%%%%%%%%%%%%%%%%%%%%
\font\defaultfn=cmr10
\font\capitolofont=cmssdc10 at 30pt
\font\sezionefont=cmssdc10 at 15pt
\font\sottosezionefont=cmssdc10 at 12pt
\font\footnotefont=cmr10 at 7pt
\font\Gotico=pxmia%<--- Definisce il Gotico
\font\Gruppi=pxsyb%<--- Definisce il carattere ''sbarretta''
%%%%%%%%%%%%%%%%%%%%%%%%%%%%%%%%%%%%%%%%%%%%%%%%%%%%%%%%%%%%%%%%%%%%%%%%%%
%	from tex-by-topic pg211
\def\goodbreak{\par\penalty-500 } % encourage page break
\def\capitolo#1\par{%
	\vfill\supereject
	\noindent\line{\capitolofont #1\hfil}%
	\vskip 2cm
}%
\def\sezione#1\par{%
	\bigskip
	\goodbreak
	\noindent\line{\sezionefont\# #1\hfil}%
	\smallskip\noindent
}%
\def\sottosezione#1\par{%
	\bigskip
	\goodbreak
	\noindent\line{\sottosezionefont #1\hfil}%
	\smallskip\noindent
}%
\newcount\numeronota
% eplain reference pg30
\let\notapiepagina=\numberedfootnote
\everyfootnote={\parindent=\defaultparindent}

% per le cose importanti
\def\evidenzia<#1>{{\bf #1}}

% per indicare gli spazi (dei messaggi, delle chiavi, etc...)
\def\spazio #1{\hbox{\Gotico #1}}
\def\gruppo #1{\hbox{\Gruppi #1}}
\def\ordinegruppo #1{$o(\gruppo #1)$}
\def\ordineelemento #1{$o(#1)$}
%%%%%%%%%%%%%%%%%%%%%%%%%%%%%%%%%%%%%%%%%%%%%%%%%%%%%%%%%%%%%%%%%%%%%%%%%%
\def\slash#1{#1\llap{/}}
%%%%%%%%%%%%%%%%%%%%%%%%%%%%%%%%%%%%%%%%%%%%%%%%%%%%%%%%%%%%%%%%%%%%%%%%%%
\newbox\littletitle%
\newdimen\littletitledim\littletitledim=1.5cm
\newdimen\littlespacedim\littlespacedim=0cm
\def\voce#1: #2\par{%
	\goodbreak
	\setbox\littletitle\hbox{\bf #1: }%
	\ifdim\wd\littletitle < \littletitledim
		% calculate how many space need to align
		\littlespacedim=\the\littletitledim
		\advance\littlespacedim by -\the\wd\littletitle
		\hangindent=\littletitledim
		\hangafter=1
		\noindent\box\littletitle\hskip\the\littlespacedim #2
		\medskip
	\else
		\noindent\line{\box\littletitle\hfil}%
		\par
		\advance\leftskip by \littletitledim
		\noindent #2%
		\medskip
		\advance\leftskip by -\littletitledim
	\fi
}%
%%%%%%%%%%%%%%%%%%%%%%%%%%%%%%%%%%%%%%%%%%%%%%%%%%%%%%%%%%%%%%%%%%%%%%%%%%
% per le citazioni
\def\citazione#1\par{
	\goodbreak
	\bgroup
	\bigskip
	\parindent=2cm\narrower\sl
	\noindent #1
	\bigskip
	\egroup
}
\def\definizione #1:#2\par{%
	\goodbreak
    \bigskip
    \line{\hrulefill}
    \line{{\bf Definizione X:} #1\hfill}
    \medskip\noindent #2\medskip
    \line{\hrulefill}
    \bigskip
}
% http://www.texample.net/tikz/examples/boxes-with-text-and-math/
% Define box and box title style
\tikzstyle{mybox} = [draw=black,fill=blue!10,
    rectangle, rounded corners, inner sep=10pt, inner ysep=20pt]
\tikzstyle{fancytitle} =[fill=gray, text=white]

\def\ddefinizione #1:#2\par{%
    \medskip
\line{\hfil\tikzpicture
\node [mybox] (box){%
    \vbox{\hsize.7\hsize #2}
};
\node[fancytitle, right=10pt] at (box.north west) {#1};
\node[fancytitle, rounded corners] at (box.east) {$\clubsuit$};
\endtikzpicture\hfil}
\medskip
}

\def\teorema #1:#2\par{%
	\goodbreak
    \bigskip
    \line{{\bf Teorema X:} #1\hfill}
    \medskip\noindent #2\medskip
    \line{\hskip 5cm\hrulefill}
    \bigskip
}
\def\corollario #1\par{%
    \bigskip\noindent{\bf Corollario X:} #1\medskip
    \bigskip
}
\def\newpage{\vfil\eject}

%%%%%%%%%%%%%%%%% TIKZ %%%%%%%%%%%%%%%%%%%
\usetikzlibrary{shapes,snakes}
\def\slashed #1{%
    \tikz[baseline=(char.base)]{%
        \node[shape=strike out,draw,inner sep=2pt] (char) {#1};
    }
}
\def\circled #1{%
    \tikz[baseline=(char.base)]{%
        \node[shape=circle,draw,inner sep=2pt] (char) {#1};
    }
}
\def\todo #1{}
