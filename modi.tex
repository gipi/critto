\capitolo Cifrari a blocchi

Volendo cifrare messaggi di arbitraria lunghezza si necessit\`a di utilizzare dei
cifrari a blocchi spezzando il messaggio in un certo numero di blocchi.

\sezione Introduzione matematica

Partiamo dal fissare la \evidenzia<dimensione del blocco>: indichiamola con $B$.
Per ottenere un algoritmo che cifri in maniera utile \`e necessario che indipendentemente
dall'input, l'output risulti il pi\`u possible casuale.

\sezione Primitive

Esistono varie metodologie per creare funzioni pseudo-random

\sottosezione Substitution-Permutation networks

\sottosezione Feistel network


\sezione Modi di operazione

\sottosezione CTR mode

\sottosezione CBC mode

\sottosezione ECB mode

