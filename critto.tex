\input macro

\capitolo Introduzione

I nomi tecnici molte volte vengono mantenuti in inglese per essere facilmente
riconoscibili e approfonditi nella letteratura standard (che sono in inglese).

\sezione Storia

Ci sono alcuni algoritmi che sono usati sin dall'antichit\`a. La loro facilit\`a nel
romperli render\`a evidenti alcuni principi esposti in seguito.

Partiamo con i tipi di cifrario pi\`u semplice, i \evidenzia<cifrari a sostituzione>:
in questi schemi una lettera viene sostituta con un'altra seconda una qualche regola.

Il pi\`u semplice \`e lo \evidenzia<shift cipher> in cui ogni lettera viene sostituita
con quella che si trova ad un certo numero di caratteri da essa, la chiave \`e proprio
l'offset. Ovviamente \`e triviale ``rompere'' questo cifrario in quanto per un normale
alfabeto le chiavi possibili sono qualche decina, quindi esauribili in un semplice
attacco di forza bruta.

Un perfezionamento \`e scegliere una permutazione delle lettere dell'alfabeto in modo
da avere uno spazio delle chiavi pari a $26!\sim 2^?$

\todo hill's cipher

\ddefinizione Principio di Kierchoffs:L'algoritmo di cifratura non deve essere segreto,
l'unico segreto deve essere la chiave che deve essere scelta in maniera casuale. Molto
importante \`e che lo spazio delle chiavi deve essere largo abbastanza da prevenire
un attacco di forza bruta.

Una cosa di cui tenere conto per valutare correttamente il principio sopra esposto \`e
il fatto che il tuo avversario pu\`o essere il tuo interlocutore con il quale utilizzi
l'algoritmo che dovrebbe rimanere segreto. Poi nella pratica, un reverser capace di leggere
assembler pu\`o comprendere il codice senza problemi\notapiepagina{In assembler qualunque
codice \`e open source.}. Inoltre si pu\`o avere pi\`u di un interlocutore e ovviamente
si dovrebbe avere un algoritmo diverso per ognuno (quindi un programma diverso).

Negli anni 70 e 80 inizia a definirsi una scienza vera e propria
\medskip
\voce Definizioni formali: utile per capire cosa si ottiene e comporre con altri schemi

\voce Assunti: assunti computazionali

\voce Dimostrazioni di sicurezza: fornire una dimostrazione rigorosa che una costruzione
soddisfa una certa definizione sotto specifici assunti

\sezione Threat model

La cosa pi\`u importante per definire se un algoritmo \`e sicuro \`e capire
cosa l'avversario pu\`o fare.

Prima di tutto defininiamo in linea di massima i tipi di attaccanti che
generalmente si definiscono per una definzione di sicurezza:
\smallskip
\voce ciphertext-only-attack: l'avversario ha a disposizione solo un testo cifrato
e deve decifrarlo

\voce known-plaintext attack: l'avversario ha a disposizione uno o pi\`u testi cifrati
con i correspettivi plaintext.

\voce chosen-plaintext-attack: l'attaccante pu\`o permettersi di sottoporre all'oracolo
plaintext di sua scelta

\voce chosen-ciphertext-attack: l'attaccante pu\`o, oltre a sottoporre all'oracolo
plaintext a sua scelta, anche di avere decifrati dei ciphertext di sua scelta.
In generale cipher che sono malleabili non resistono a questi attacchi.

Ovviamente sono in ordine di potenza, con i primi due esclusivamente passivi.

Oltre a queste considerazioni bisogna tenere conto alcuni aspetti particolari della crittografia:
spesse volte il nostro avversario non pu\`o essere per forza di cose definito con precisione,
in quanto viene richiesto di resistere per 20 anni ad attacchi e ovviamente non possiamo avere
una idea precisa delle possibilit\`a future. Quindi \`e importante, nel prgettare sistemi che
utilizzano primitive crittografiche, fare in modo di limitare i danni in caso di failure di
altre porzioni del sistema.

\sezione Perfect secrecy

\citazione Nonostante ogni informazione precedente che l'attaccante ha riguardo al plaintext
il ciphertext non deve fornire nessuna informazione aggiuntiva su di esso.

Sia $M$ una variabile random che indica il valore di un messaggio
Formalmente, uno schema di cifratura con spazio dei messaggi \spazio M e spazio dei
testi cifrati \spazio C \`e perfettamente sicuro se per ogni distribuzione su \spazio M
si ha che
$$
\hbox{Pr}\left[M=m|C=c\right] = \hbox{Pr}\left[M=m\right]
$$

\input modi

\input random

\input autenticazione

\input pratici

\input appendice

\end
