\input macro

\readtocfile

\capitolo Introduzione

I nomi tecnici molte volte vengono mantenuti in inglese per essere facilmente
riconoscibili e approfonditi nella letteratura standard (che sono in inglese).

\sezione Storia

Ci sono alcuni algoritmi che sono usati sin dall'antichit\`a. La loro facilit\`a nel
romperli render\`a evidenti alcuni principi esposti in seguito.

Partiamo con i tipi di cifrario pi\`u semplice, i \evidenzia<cifrari a sostituzione>:
in questi schemi una lettera viene sostituta con un'altra seconda una qualche regola.
Il pi\`u semplice \`e lo \evidenzia<shift cipher> in cui ogni lettera viene sostituita
con quella che si trova ad un certo numero di caratteri da essa, la chiave \`e proprio
l'offset. Ovviamente \`e triviale ``rompere'' questo cifrario in quanto per un normale
alfabeto le chiavi possibili sono qualche decina, quindi esauribili in un semplice
attacco di forza bruta.

Un perfezionamento \`e scegliere una permutazione delle lettere dell'alfabeto in modo
da avere uno spazio delle chiavi pari a $26!\sim 2^?$

\todo hill's cipher

\ddefinizione Principio di Kierchoffs:L'algoritmo di cifratura non deve essere segreto,
l'unico segreto deve essere la chiave che deve essere scelta in maniera casuale. Molto
importante \`e che lo spazio delle chiavi deve essere largo abbastanza da prevenire
un attacco di forza bruta.

Una cosa di cui tenere conto per valutare correttamente il principio sopra esposto \`e
il fatto che il tuo avversario pu\`o essere il tuo interlocutore con il quale utilizzi
l'algoritmo che dovrebbe rimanere segreto. Poi nella pratica, un reverser capace di leggere
assembler pu\`o comprendere il codice senza problemi\notapiepagina{In assembler qualunque
codice \`e open source.}. Inoltre si pu\`o avere pi\`u di un interlocutore e ovviamente
si dovrebbe avere un algoritmo diverso per ognuno (quindi un programma diverso).

Negli anni 70 e 80 inizia a definirsi una scienza vera e propria
\medskip
\voce Definizioni formali: utile per capire cosa si ottiene e comporre con altri schemi

\voce Assunti: assunti computazionali

\voce Dimostrazioni di sicurezza: fornire una dimostrazione rigorosa che una costruzione
soddisfa una certa definizione sotto specifici assunti

\sezione Threat model

La cosa pi\`u importante per definire se un algoritmo \`e sicuro \`e capire
cosa l'avversario pu\`o fare.

Prima di tutto defininiamo in linea di massima i tipi di attaccanti che
generalmente si definiscono per una definzione di sicurezza:
\smallskip
\voce ciphertext-only-attack: l'avversario ha a disposizione solo un testo cifrato
e deve decifrarlo

\voce known-plaintext attack: l'avversario ha a disposizione uno o pi\`u testi cifrati
con i correspettivi plaintext.

\voce chosen-plaintext-attack: l'attaccante pu\`o permettersi di sottoporre all'oracolo
plaintext di sua scelta

\voce chosen-ciphertext-attack: l'attaccante pu\`o, oltre a sottoporre all'oracolo
plaintext a sua scelta, anche di avere decifrati dei ciphertext di sua scelta.
In generale cipher che sono malleabili non resistono a questi attacchi.

Ovviamente sono in ordine di potenza, con i primi due esclusivamente passivi.

Oltre a queste considerazioni bisogna tenere conto alcuni aspetti particolari della crittografia:
spesse volte il nostro avversario non pu\`o essere per forza di cose definito con precisione,
in quanto viene richiesto di resistere per 20 anni ad attacchi e ovviamente non possiamo avere
una idea precisa delle possibilit\`a future. Quindi \`e importante, nel prgettare sistemi che
utilizzano primitive crittografiche, fare in modo di limitare i danni in caso di failure di
altre porzioni del sistema.

\sezione Encryption schemes

Il soggetto principale di questi appunti è un \evidenzia<encryption scheme>, cio\`e un sistema
di cifratura che a partire da un certo tipo di messaggi, possa trasformarli e ``renderli illeggibili''.

Il certo tipo di messaggi \`e definito dallo spazio \spazio M, che ha una distribuzione probabilistica
dei suoi elementi data dal contesto (si pensi ad un linguaggio naturale con la distribuzione di probabilit\`a
di parole e lettere etc...). La probabilit\`a di un messaggio $m$ \`e data da $\Pr[\gruppo M = m]$.

\definizione ES: Definiamo uno schema di cifratura $\Pi$, una tripla di algoritmi $(\algoritmo Gen , \algoritmo Enc , \algoritmo Dec )$
tali che
\unorderedlist
\li \algoritmo Gen : algoritmo di generazione della chiave, pu\`o essere stocastico, si pu\`o dimostrare
che pu\`o essere scelta in maniera uniforme. Essa definisce $\Pr[\spazio K=k]$.
\li \algoritmo Enc : algoritmo di cifratura vero e proprio, pu\`o essere non deterministico
\li \algoritmo Dec : algoritmo ``inverso'' di \algoritmo Enc
\endunorderedlist

\sezione Perfect secrecy

\citazione Nonostante ogni informazione precedente che l'attaccante ha riguardo al plaintext
il ciphertext non deve fornire nessuna informazione aggiuntiva su di esso.

Sia $M$ una variabile random che indica il valore di un messaggio
Formalmente, uno schema di cifratura con spazio dei messaggi \spazio M e spazio dei
testi cifrati \spazio C \`e perfettamente sicuro se per ogni distribuzione su \spazio M
si ha che, per ogni messaggio $m$
$$
\Pr[M=m\,|\,C=c] = \Pr[M=m]
$$
cio\'e conoscere il testo cifrato non d\`a nessuna informazione al riguardo del messaggio da decifrare.

Si pu\`o dimostrare che questo \`e equivalente a chiedere che per ogni coppia di messaggi $m$ e $m^\prime$
$$
\Pr[{\algoritmo Enc }_K(m^\prime) = c] = \Pr[{\algoritmo Enc }_K(m) = c]
$$
dove la probabilit\`a \`e presa rispetto alla distribuzione di \spazio K.

Una cosa importante da tenere a mente riguardo la notazione che pu\`o sembrare fuorviante: definiamo proprio
$\Pr[\spazio C=c\,|\,\spazio M=m] = \Pr[\ENC\spazio K(m) = c]$, per comodit\`a, siccome l'algoritmo di cifratura
\`e scritto naturalmente a partire da un messaggio in \spazio M, per generare un elemento in \spazio C.

Per ricavare la relazione pi\`u di nostro interesse utilizziamo la regola di Bayes cos\`\i\ da ottenere
$$
\Pr[\spazio M=m\,|\,\spazio C=c] = {\Pr[\spazio C=c\,|\,\spazio M=m]\Pr[\spazio M=m]\over\Pr[\spazio C=c]}
$$
Le due relazioni sono uguali solo se $\Pr[\spazio M=m] = \Pr[\spazio C=c]$, che come vedremo pu\`o ritenersi vero
nei casi applicabili al teorema di Shannon.

\input modi

\input random

\input autenticazione

\input pratici

\appendice A

Nell'appendice saranno inserite le cose pi\`u formali dal punto di vista matematico
che sono fini a se stesse.

Partiamo con definire un \evidenzia<insieme> (che indicheremo con simboli del tipo \spazio A, \spazio B, \spazio C)
come una collezione di elementi senza una struttura sottostante.

\input appendice

\end
